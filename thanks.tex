\section*{Remerciements}
\begin{footnotesize}

Je tiens à adresser ci-après mes remerciements aux personnes qui ont permis l'aboutissement de ce rapport, de près ou de loin, et qui ont contribué de façon plus ou moins directe à mes travaux depuis Septembre.

à Nicolas VERGNE, qui a su lever des doutes et répondre à mes anxiétés (parfois infondées) lors de nos rendez-vous à Rouen.

À l'équipe ``Bactéries Alimentaires et Commensales'' (BAC) pour son accueil agréable ainsi qu'aux collègues de l'étage.
Leur patience à m'entendre parler de cuisine régulièrement est louable et ils seront remerciés à mon pot de départ !

À Hugo DEVILLERS, intermittent du bureau (et futur représentant \textsc{Bodum}), pour nos discussions geeks et ses retours d'expérience académiques, autour d'un café bien sûr.

À Thibaut GUIRIMAND, pour ses blagues derrière moi, littéralement, et sans qui les relations entre "collègues" seraient différentes.

À Mathieu ALMEIDA, pour ses suggestions intéressantes et les folles conversations à chacune de ses visites au laboratoire. 

À Christine DELORME et Éric GUÉDON pour leurs questions pertinentes lors de mes présentations.

à Sophie SCHBATH sans qui mes premières modélisations n'auraient pas dépassé un stade embryonnaire.

Aux personnes rencontrées à JOBIM, face au poster et surtout ailleurs, pour les discussions re-dynamisantes.

À la communauté du logiciel libre, sans qui de nombreux outils performants n'existeraient pas pour tous.

À mon ``bro'' de trail, grâce à qui j'ai pu lever la tête du guidon et m'aérer les méninges en forêt cette année.

À Pierre RENAULT, pour ces fascinantes histoires sur la diversité de ses chers micro-organismes et les discussions scientifiques qui suivent.

À Mahendra MARIADASSOU pour ses conseils avisés et ses mots justes, toujours, malgré mes questions bêtes. Sa clarté de formulation des problèmes et l'élégance de ses solutions est stimulante.

À Anne-Laure ABRAHAM, dont j'ai parfois maudit le stylo correcteur, mais en appréciant toujours la justesse de ses remarques et son art de la reformulation claire. Sa patience quotidienne face à mes digressions et bavardages est désormais légendaire.  Lorsque je serais à moitié aussi bien organisé qu'elle, ce sera un accomplissement personnel.

Ces relecteurs méritent aussi un gâteau au vu du harcélement à des heures indues dont j'ai pu faire preuve.

Aux personnes qui m'entourent et avec qui je partage stress et déboires tout autant qu'excitation et succès.

Aux optimistes, sans qui on aurait tout arrêté sans doute, mais grâce à qui l'on continue.

À toi, lecteur lectrice, qui est au début de l'histoire.
\end{footnotesize}
