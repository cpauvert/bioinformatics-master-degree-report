\section*{Cheese ecosystems
metagenomics}\label{cheese-ecosystems-metagenomics}
\vspace{-2em}
\subsubsection*{Exploration \& improvements around a bioinformatics
tool}\label{exploration-improvements-around-a-bioinformatics-tool}

Cheese complex flora --composed of dairy micro-organisms-- is not
completely and exactly known in most cheeses. Further understandings of
these ecosystems needs a better characterization of the cheese flora and
a precise taxonomic identification. Hundreds of genomes extracted from
dairy products are currently available in genomics databases. But the
key points to tackle are low abundant species and identification up to
the strain level. A tool was developed in the team to address these
issues using shotgun metagenomics sequencing data.

I have focused my research around this tool for two years. I mostly
worked on scientific improvements: deeply exploring new leads to enhance
ecosystem taxonomic identification. Notably I designed an approach
taking into account the discrepancy between reference genomes and
genomes from the environment. However, it did not yields expected
results despite a thorough examination of model limits.

I have also worked on computational features improvements such as
compatibility by using standard bioinformatics files. This conversion
leads to space usage and speed improvements.\\
I am involved in the integration of this tool with a metadata-extended
genomics database.

\paragraph{Keywords} metagenomics, microbial community, simulated datasets.
\vspace{-2em}
\paragraph{Technical keywords} short reads alignement, CDS coverage,
mixture model.

\section*{Métagénomique des écosystèmes
fromagers}\label{muxe9taguxe9nomique-des-uxe9cosystuxe8mes-fromagers}
\vspace{-2em}
\subsubsection*{Améliorations et explorations d'un outil
bioinformatique}\label{amuxe9liorations-et-explorations-dun-outil-bioinformatique}

La composition exacte des flores fromagères n'est pas connue dans la
plupart des fromages. Afin d'approfondir les connaissances de ces
écosystèmes, une meilleure caractérisation de la flore fromagère et une
assignation taxonomique précise sont nécessaires. Quelques centaines de
génomes issus de produits laitiers sont désormais disponibles dans les
banques de données. Cependant, la faible abondance de certaines espèces
et l'identification jusqu'à la souche restent des défis. Dans l'équipe,
un outil a été développé pour les aborder à partir de données de
séquençage métagénomique global aléatoire.

Durant deux ans, j'ai focalisé mes travaux autour de cet outil. J'ai
principalement travaillé à des perfectionnements telles que
l'exploration de nouvelles pistes pour améliorer l'identification de
micro-organismes. J'ai développé une approche qui implique la
discordance entre génomes de références et génomes de l'environnement.
Malgré une exploration exhaustive des limites du modèles, cette méthode
n'offre cependant pas les résultats escomptés.

J'ai également travaillé à améliorer les caractéristiques de l'outil
telles que l'inter-opérabilité en utilisant des standards en
bioinformatique. Cette conversion a permis de limiter l'espace disque
utilisé et de diminuer le temps d'éxecution. J'ai également été impliqué
dans l'intégration de cet outil avec une base de données génomique
enrichies en métadonnées.

\paragraph{Mots-clés} métagénomique, microbiome, alignement de courtes
lectures.
\vspace{-2em}
\paragraph{Mots-clés techniques} données simulées, couvertures des CDS,
modèle de mélange.
